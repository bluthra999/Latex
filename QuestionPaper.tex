\documentclass{book}
\usepackage{anysize}
\usepackage{amsmath}
\usepackage{mathabx}
\usepackage{amstext}
\usepackage{enumitem}
\usepackage{lmodern}
\usepackage{setspace}
\usepackage[paperheight=7.7in,paperwidth=5.5in]{geometry}

% Custom header
\usepackage{fancyhdr}
\pagestyle{fancy}
\fancyhf{}
\lhead{\textbf{7568}}
\chead{\textbf{\thepage}}
\renewcommand{\headrulewidth}{0pt}
\fancyfoot[RO]{P.T.O}

\begin{document}

\parindent=0pt
\parskip=8pt

[This question paper contains 6 printed pages.]

\hfill{\textbf{Your Roll No.............}}

\textbf{Sr. No. of Question Paper: 7568} \hfill \textbf{J}

\begin{tabular}{ll}
Unique  Paper Code &: \hspace{12pt} 32223902\\[8pt]
Name of the Paper &: \hspace{12pt} Computational Physics Skills\\[8pt]
Name of the Course &:
\begin{tabular}{rl}
& \textbf{B.Sc. (Hons.) Physics}\\
& \textbf{B.Sc. (Prog.): SEC}
\end{tabular}\\[8pt]

Semester &: \hspace{12pt} III\\[8pt]
\end{tabular}

\hspace{5pt}Duration : 3 Hours \hfill Maximum Marks : 50

\textbf{\underline{Instructions for Candidates}}
\begin{enumerate}
    \item  Write your Roll No. on the top immediately on receipt of this question paper.
    \item Question No. 1 is compulsory.
    \item Attempt 3 questions from each section.
\end{enumerate}

\onehalfspacing

\begin{enumerate}
\setlength\itemsep{1em}

\item  Attempt any five questions : \hfill (5x1=5)

\begin{enumerate}[label=(\alph*)]
\setlength\itemsep{0.75em}

\item Describe the fortran statement parameter (pi=\\ 3.14159).

\item  Give example (without explanation) of the fortran\\
statements required to open a file for data writing, \\
writing data in the opened file, closing the opened \\
data file.

\item What is meant by a structured programming \\ language?

\item Briefly describe the use of any two LaTeX packages.

\item What is a documentclass in LaTeX? Name some \\ documentclasses.

\item write the gnuplot statements to put title on the \\ graph and labels on the axes.

\item Write the gnuplot statements to define a function \\ (2X+1) and plot it.

\end{enumerate}

\vspace{1em}
\begin{center} \textbf{SECTION - A} \end{center}


\item  Write a fortran program to multiply (m x k) matrix \\
with (k x n) matrix to give an m x n matrix. \hfill (5)

\item Write a fortran FUNCTION to calculate factorial \\
of a number. Use this FUNCTION in a fortran. program to evaluate the value of
$\cos(x) = 1 -\frac{x^2}{2!} + \frac{x^4}{4!}$
\\ at x = 0.5 \hfill (5)

\item Write the syntax of logical if, arithmetic if and block \\
if statements in fortran. \hfill (5)

\item Write a fortran program to evaluate roots of a \\
quadratic equation considering all three cases where \\
discriminant (D) : D = 0, D $>$ 0, D $<$ 0. \hfill (5)

\vspace{1em}
\begin{center} \textbf{SECTION - B} \end{center}

\item Write a LaTeX code to display the following Maxwell \\ equations. \hfill (5)

    \begin{align*}
        &\nabla \cdot \mathbf{D} = {\rho}\\
        &\nabla \cdot \mathbf{B} = 0  \\
        &\nabla \times \mathbf{E} =-\frac{\partial \mathbf{B}}{\partial t}\\
        &\nabla \times \mathbf{H} =\mathbf{J}+ \frac{\partial \mathbf{D} }{\partial t}
    \end{align*}

\item Write the output of the following Latex code

{\fontsize{8.88}{1pt}\selectfont{
\begin{verbatim}
\[\scalebox{3} {$ I=I_{0} \left(e^{\frac {e V} {\eta K
T}}-1\right)$} \]                                  (5)
\end{verbatim}
}}


\item Write the Latex code to generate following equation
$$
I = I_0
    \left(
        \frac{\sin\left( \frac{\pi a \sin\theta}{\lambda}\right)}{\frac{\pi a \sin\theta}{\lambda}}
    \right)^2
    \left(
        \frac{\sin\left( \frac{N\pi d \sin\theta}{\lambda}\right)}{\sin\left(\frac{\pi d \sin\theta}{\lambda}\right)}
    \right)^2
$$

This equation describes the Intensity distribution \\
resulting from interferense of diffraction patterns of \\
N-slits. Here a = slit width, d= distance between \\
centers of two consecutive slits( grating element), $\theta$ \\
is the angle of diffraction and A is the wavelength of \\
light used. \hfill (5)



\item Describe any method of including bibliography and \\
 citations in a LaTeX document. Also describe any \\
 method to include index in the LaTeX document.\\
  \phantom{.} \hfill (5)



  \vspace{1em}
  \begin{center} \textbf{SECTION - C} \end{center}


 \item Describe briefly any five Terminals and corresponding \\
 output file extenstions .xxx in gnuplot that can be set \\
 using set terminal terminal-type command and set out \\
"filename.xxx" command. \hfill (5)


\item Describe the use of any three of the following gnuplot statements

\begin{enumerate}[label=(\alph*)]
\setlength\itemsep{1em}
\item "save" and "load"
 \item set parametric
 \item set pm3d
 \item set samples 3000
 \item set title
 \item unset key \hfill (5)

\end{enumerate}


    \item Using parametric curve plotting in gnuplot, draw the \\
    trajectory of a projectile fired with initial velocity 100 \\
     m/s at an angle of 45° with the horizontal. \hfill (5)
    \item Describe the outcome of the following gnuplot script

    reset

    set multiplot layout 2,2 columns first scale 1,1

    plot sin(x)

    plot cos(x)

    plot x**3

    plot sin(x)**2

    unset multiplot \hfill (5)

\setstretch{0.75}
\item  Given the functions

\begin{flalign*}
\text{fl(x)}=\sin(\pi x) &&
\end{flalign*}

\begin{flalign*}
    \text{f2(x)}=\frac{\sin(3\pi x)}{3}&&
\end{flalign*}

\begin{flalign*}
\text{f3(x)}=\frac{\sin(5\pi x)}{5}&&
\end{flalign*}

\begin{flalign*}
\text{f4(x)}=\frac{\sin(7\pi x)}{7}&&
\end{flalign*}

\singlespacing
write a gnuplot script to plot (f1(x) - f2(x) + f3(x) -

f4(x)) as a function of x. \hfill (5)


\end{enumerate}

% Change  footer on the last page
\rfoot{\textbf{(800)}}

\end{document}